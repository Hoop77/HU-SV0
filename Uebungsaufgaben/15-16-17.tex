\documentclass[fleqn,a4paper,12pt]{article}

%used Packages
\usepackage{standalone}		% Zum Einlesen aus anderen .tex-Files
\usepackage{geometry}		% Zur Bearbeitung des Layouts (Ränder,...)
\usepackage[german]{babel}
\usepackage[utf8]{inputenc}
\usepackage{amsmath}		% Mathematische Symbole
\usepackage{amssymb}     	% Nochmehr mathematische Symbole
\usepackage{dsfont}      	% Schriftsatz fuer Zahlenmengensymbole
%\usepackage{verbatim}   	% erweiterte Verbatim-Umgebung
\usepackage{alltt}       	% Quasi-Verbatim-Umgebung
\usepackage{fancyhdr}    	% Eigene Kopfzeilen
\usepackage{graphicx}    	% Zum Einbinden von Grafiken
							% Einbinden einer eps-Grafik geht so: includegraphics{path}
\usepackage{wrapfig}
\usepackage{lscape}
\usepackage{rotating}
\usepackage{epstopdf}

% Skalierung der Grafiken
\setlength{\unitlength}{1cm}

\frenchspacing               % Kein Extrafreiraum nach Satzzeichen
\setlength{\parindent}{0pt}  % Neue Absaetze nicht einruecken
%\sloppy                     % Schlampige Absatzformatierung
\fussy                       % Penible Absatzformatierung
\linespread{1.5}             % Zeilenabstand


% Seitenraender
\geometry{left=30mm, right=40mm, bottom=30mm}
				% Doc-class, Packageimports, fancy stuff
%Seitenränder formatieren
\addtolength{\voffset}{-2cm}
\addtolength{\textheight}{0cm}
\addtolength{\hoffset}{0cm}
\addtolength{\textwidth}{2cm}
\addtolength{\headheight}{2cm} % fuer jeden Strichkode einen Zentimeter

% Font fuer Code 39
\font\xlix=wlc39 scaled 1200
\newcommand\barcode[1]{{\xlix@#1@}}

% Name, Matrikelnummer, Barcode
\newcommand\student[2]{
	\mbox{\scriptsize
		\begin{tabular}{@{}l@{}r@{}}
			\multicolumn{2}{@{}r@{}}{\barcode{#2}}\\
			#1&#2\\
		\end{tabular}}}

% Kopfzeile
\pagestyle{fancy}            % Eigene Kopfzeilen verwenden
\lhead{
	\small
	\textsc{Grundlagen der Signalverarbeitung \\
		WS 2017/2018 \\
		\"Ubung (\today)}
	\vfill}
\rhead{
	\begin{tabular}[b]{@{}rr@{}}
		\student{Philipp Badenhoop}{572693} &
		\student{Steven Lange}{568733} \\
		\student{Pascal Jochmann}{575056} &
		\student{Kevin Trogant}{572451}
\end{tabular}}			% Definition der Kopfzeile
%andere Definitionen
\providecommand{\R}{{\mathbb R}}
\providecommand{\N}{{\mathbb N}}
\providecommand{\Z}{{\mathbb Z}}
\providecommand{\Q}{{\mathbb Q}}
\providecommand{\C}{{\mathbb C}}
\providecommand{\F}{\mathcal{F}}
\providecommand{\less}{\setminus}
\providecommand{\inv}{{}^{-1}}
\providecommand{\Land}{\bigwedge}
\providecommand{\Lor}{\bigvee}			% Liste der zusätzlichen Commands und redefines

\begin{document}
	\section*{Übungsaufgabe 15:}

	Es liegt ein Polynom mit mindestens Grad 4 vor. Man erkennt in Abbildung~\ref{fig:wartezeiten}, dass das Polynom 3. Grades einen zu starken Anstieg
	besitzt, um die 80 Minuten Wartezeit von Besucher 941 zu ``treffen''. Anscheinend war der größte Andrang an Besuchern bereits vor Beuscher 941.

	\begin{sidewaysfigure}
        \includegraphics[width=1.0\textwidth]{A15.png}
        \caption{(15) Wartezeiten}
        \label{fig:wartezeiten}
    \end{sidewaysfigure}
	
\newpage
    \section*{Übungsaufgabe 16}
    Gegebene Messwerte: \\
    \begin{center}
    \begin{tabular}{|c|c|c|}
        \hline 
        $n$ & $t_n$ & $f_n$ \\
        \hline
        0   &  0    & 3 \\
        1   &  1    & 3 \\
        2   &  3    & 9 \\
        3   &  4    & 15 \\
        \hline
    \end{tabular}
    \end{center}
    Wir wollen $f_{ap}(t) = c_1 t + c_0 t^0$ bestimmen. Aufstellen des Gleichungssystems liefert:
    \begin{align*}
        c_0 \sum_{n=0}^{3} t_n^0 t_n^0 + c_1 \sum_{n=0}^3 t_n^0 t_n^1 &= \sum_{n=0}^3 f_n t_n^0 \\
        \quad c_0 \sum_{n=0}^{3} t_n^1 t_n^0 + c_1 \sum_{n=0}^3 t_n^1 t_n^1 &= \sum_{n=0}^3 f_n t_n^1
    \end{align*}
    Einsetzen liefert:
    \begin{align*}
        4c_0 + 8c_1 &= 30 \\
        8c_0 + 26c_1 &= 90
    \end{align*}
    Einsetzen der Randbedingung $c_0 = 3$ liefert:
    \begin{align*}
        8c_1 &= 18  \Rightarrow c_1 = \frac{9}{4} \quad \textbf{Lsg 1} \\
        26c_1 &= 66 \Rightarrow c_1 = \frac{33}{13} \quad \textbf{Lsg 2}
    \end{align*}
    Für $c_0 = 3, c_1 = \frac{9}{4}$ ergibt sich:
    \begin{align*}
        E^2(c) = \sum_{n=0}^3 \left[ f_n - 3t_n^0 - \frac{9}{4}t_n^1 \right]^2 = \frac{117}{8} = 14.625
    \end{align*}
    Für $c_0 = 3, c_1 = \frac{33}{13}$ ergibt sich:
    \begin{align*}
        E^2(c) = \sum_{n=0}^3 \left[ f_n - 3t_n^0 - \frac{33}{13}t_n^1 \right]^2 = \frac{162}{13} \approx 12.462 
    \end{align*}
    \begin{figure}
        \includegraphics[width=\textwidth]{A16_plot.png}
        \caption{Grafische Darstellung}
    \end{figure}
\newpage
	\section*{Übungsaufgabe 17:}
		Sei $f(t) = t^2-t +3$ mit $t\in[0,4]$. Sei weiter $F_{app}(t) = c_1t+3$.
		\begin{align*}
			e^2(c)	&= \int_0^4 \left[ f(t)-f_{app}(c,t) \right]^2 dt\\
					&= \int_0^4 \left[ t^2 -t(1+c_1) \right]^2 dt\\
					&= \int_0^4 t^4 -2t^3(1+c_1) + t^2(1+c_1)^2 dt\\
					&= \left.\frac{t^5}{5}\right|_0^4 - \left.\frac{t^4}{2}(1+c_1)\right|_0^4 + \left.\frac{t^3}{3}(1+c_1)^2 \right|_0^4\\
					&= \frac{1472}{15} - \frac{256}{3}c_1 + \frac{64}{3}c_1^2\\
					&\\
			0\overset{!}{=} \frac{d}{dc_1} e^2(c)	&= -\frac{256}{3} + \frac{128}{3}c_1\\
					\Leftrightarrow	\frac{256}{3}	&= \frac{128}{3}c_1\\
					\Leftrightarrow				c_1	&= 2
		\end{align*}
		Das heißt, dass $e^2(c)$ für $c_1 = 2$ mit $e^2(c) = \frac{1572}{3}\approx 98.13$ minimal wird.\\
		\\
		Der Unterschied zu Aufgabe 16 besteht daran, dass nicht ein beliebiges Set aus $[0,4]$ gewählt, sondern unendlich viele (mit anderen Worten alle) Paare $(t_n, f_n)$ gewählt werden um den Fehler zu bestimmen. Die 16 selbst würde mit einem randomisierten neuen Set aus 4 Paaren und selber Randbedingung einen anderen Wert ausgeben.
		
		\includegraphics[scale = 0.7]{A17_functionPlot.png}
\newpage\end{document}