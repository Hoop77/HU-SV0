\documentclass[fleqn,a4paper,12pt]{article}
\usepackage{standalone}		% Zum Einlesen aus anderen .tex-Files
\usepackage{geometry}		% Zur Bearbeitung des Layouts (Ränder,...)
\geometry{left=30mm, right=40mm, bottom=30mm}
\usepackage[german]{babel}
\usepackage[utf8]{inputenc}
\usepackage{amsmath}		% Mathematische Symbole
\usepackage{amssymb}     	% Nochmehr mathematische Symbole
\usepackage{dsfont}      	% Schriftsatz fuer Zahlenmengensymbole
%\usepackage{verbatim}   	% erweiterte Verbatim-Umgebung
\usepackage{alltt}       	% Quasi-Verbatim-Umgebung
\usepackage{fancyhdr}    	% Eigene Kopfzeilen
\usepackage{graphicx}    	% Zum Einbinden von Grafiken
% Einbinden einer eps-Grafik geht so: includegraphics{path}
\usepackage{wrapfig}
\usepackage{lscape}
\usepackage{rotating}
\usepackage{epstopdf}

% Seitenraender
\addtolength{\voffset}{-2cm}
\addtolength{\textheight}{0cm}
\addtolength{\hoffset}{0cm}
\addtolength{\textwidth}{2cm}
\addtolength{\headheight}{2cm} % fuer jeden Strichkode einen Zentimeter

% Skalierung der Grafiken
\setlength{\unitlength}{1cm}

\pagestyle{fancy}            % Eigene Kopfzeilen verwenden
\frenchspacing               % Kein Extrafreiraum nach Satzzeichen
\setlength{\parindent}{0pt}  % Neue Absaetze nicht einruecken
%\sloppy                     % Schlampige Absatzformatierung
\fussy                       % Penible Absatzformatierung
\linespread{1.5}             % Zeilenabstand

% Font fuer Code 39
\font\xlix=wlc39 scaled 1200
\newcommand\barcode[1]{{\xlix@#1@}}

% Name, Matrikelnummer, Barcode
\newcommand\student[2]{
\mbox{\scriptsize
	\begin{tabular}{@{}l@{}r@{}}
		\multicolumn{2}{@{}r@{}}{\barcode{#2}}\\
		#1&#2\\
	\end{tabular}}}

%andere Definitionen
\newcommand{\R}{{\mathbb R}}
\newcommand{\N}{{\mathbb N}}
\newcommand{\Z}{{\mathbb Z}}
\newcommand{\Q}{{\mathbb Q}}
\newcommand{\C}{{\mathbb C}}
\newcommand{\F}{\mathcal{F}}
\newcommand{\less}{\setminus}
\newcommand{\inv}{{}^{-1}}
\newcommand{\Land}{\bigwedge}
\newcommand{\Lor}{\bigvee}

% Kopfzeile
\lhead{
	\small
	\textsc{Grundlagen der Signalverarbeitung \\
		WS 2017/2018 \\
		\"Ubung (\today)}
	\vfill}
\rhead{
	\begin{tabular}[b]{@{}rr@{}}
		\student{Philipp Badenhoop}{572693} &
		\student{Steven Lange}{568733} \\
		\student{Pascal Jochmann}{575056} &
		\student{Kevin Trogant}{572451}
	\end{tabular}}

\begin{document}
	\documentclass[fleqn,a4paper,12pt]{article}

%used Packages
\usepackage{standalone}		% Zum Einlesen aus anderen .tex-Files
\usepackage{geometry}		% Zur Bearbeitung des Layouts (Ränder,...)
\usepackage[german]{babel}
\usepackage[utf8]{inputenc}
\usepackage{amsmath}		% Mathematische Symbole
\usepackage{amssymb}     	% Nochmehr mathematische Symbole
\usepackage{dsfont}      	% Schriftsatz fuer Zahlenmengensymbole
%\usepackage{verbatim}   	% erweiterte Verbatim-Umgebung
\usepackage{alltt}       	% Quasi-Verbatim-Umgebung
\usepackage{fancyhdr}    	% Eigene Kopfzeilen
\usepackage{graphicx}    	% Zum Einbinden von Grafiken
							% Einbinden einer eps-Grafik geht so: includegraphics{path}
\usepackage{wrapfig}
\usepackage{lscape}
\usepackage{rotating}
\usepackage{epstopdf}

% Skalierung der Grafiken
\setlength{\unitlength}{1cm}

\frenchspacing               % Kein Extrafreiraum nach Satzzeichen
\setlength{\parindent}{0pt}  % Neue Absaetze nicht einruecken
%\sloppy                     % Schlampige Absatzformatierung
\fussy                       % Penible Absatzformatierung
\linespread{1.5}             % Zeilenabstand


% Seitenraender
\geometry{left=30mm, right=40mm, bottom=30mm}
				% Doc-class, Packageimports, fancy stuff
%%Seitenränder formatieren
\addtolength{\voffset}{-2cm}
\addtolength{\textheight}{0cm}
\addtolength{\hoffset}{0cm}
\addtolength{\textwidth}{2cm}
\addtolength{\headheight}{2cm} % fuer jeden Strichkode einen Zentimeter

% Font fuer Code 39
\font\xlix=wlc39 scaled 1200
\newcommand\barcode[1]{{\xlix@#1@}}

% Name, Matrikelnummer, Barcode
\newcommand\student[2]{
	\mbox{\scriptsize
		\begin{tabular}{@{}l@{}r@{}}
			\multicolumn{2}{@{}r@{}}{\barcode{#2}}\\
			#1&#2\\
		\end{tabular}}}

% Kopfzeile
\pagestyle{fancy}            % Eigene Kopfzeilen verwenden
\lhead{
	\small
	\textsc{Grundlagen der Signalverarbeitung \\
		WS 2017/2018 \\
		\"Ubung (\today)}
	\vfill}
\rhead{
	\begin{tabular}[b]{@{}rr@{}}
		\student{Philipp Badenhoop}{572693} &
		\student{Steven Lange}{568733} \\
		\student{Pascal Jochmann}{575056} &
		\student{Kevin Trogant}{572451}
\end{tabular}}			% Definition der Kopfzeile
%andere Definitionen
\providecommand{\R}{{\mathbb R}}
\providecommand{\N}{{\mathbb N}}
\providecommand{\Z}{{\mathbb Z}}
\providecommand{\Q}{{\mathbb Q}}
\providecommand{\C}{{\mathbb C}}
\providecommand{\F}{\mathcal{F}}
\providecommand{\less}{\setminus}
\providecommand{\inv}{{}^{-1}}
\providecommand{\Land}{\bigwedge}
\providecommand{\Lor}{\bigvee}			% Liste der zusätzlichen Commands und redefines

\begin{document}
	  "Ubungsaufgabe 3: \newline
	Messreihe 1000 Würfe mit einem 6-seitigen Würfel: \newline
	5, 4, 2, 4, 5, 6, 3, 2, 5, 3, 4, 2, 1, 1, 5, 1, 1, 2, 1, 6, 4, 6, 1, 4, 5, 3, 2, 3, 2, 1, 1, 2, 6, 2, 5, 2, 5, 6, 4, 6, 4, 5, 2, 2, 3, 1, 2, 5, 5, 4, 5, 5, 4, 5, 2, 2, 5, 4, 3, 5, 5, 4, 2, 4, 3, 1, 5, 4, 1, 6, 5, 4, 5, 4, 2, 1, 4, 1, 2, 6, 1, 1, 5, 3, 5, 3, 1, 6, 4, 6, 1, 6, 2, 5, 4, 4, 2, 6, 1, 3, 4, 2, 2, 5, 3, 3, 2, 5, 6, 2, 3, 3, 3, 6, 3, 1, 4, 2, 1, 5, 6, 6, 3, 2, 3, 3, 5, 6, 3, 1, 4, 5, 6, 3, 1, 1, 4, 3, 1, 2, 3, 6, 3, 2, 3, 4, 2, 4, 6, 4, 3, 4, 3, 2, 4, 5, 5, 4, 3, 3, 2, 4, 4, 5, 5, 6, 4, 2, 6, 6, 5, 5, 6, 6, 2, 3, 1, 2, 3, 2, 5, 1, 3, 3, 2, 4, 5, 2, 4, 5, 3, 6, 5, 3, 5, 5, 2, 6, 4, 3, 6, 5, 5, 2, 1, 5, 5, 2, 1, 3, 2, 5, 3, 2, 2, 6, 3, 5, 4, 3, 1, 1, 5, 6, 5, 6, 3, 2, 3, 1, 6, 6, 3, 5, 3, 2, 5, 3, 2, 5, 2, 5, 3, 5, 2, 2, 1, 5, 4, 3, 6, 6, 2, 6, 3, 2, 3, 2, 4, 5, 4, 6, 3, 1, 3, 1, 3, 4, 1, 1, 1, 6, 2, 1, 5, 4, 5, 6, 2, 4, 1, 4, 2, 6, 3, 6, 1, 2, 2, 2, 3, 5, 3, 6, 2, 5, 2, 6, 5, 1, 2, 2, 5, 4, 2, 5, 6, 3, 1, 3, 1, 3, 6, 5, 3, 5, 2, 6, 4, 3, 4, 2, 2, 6, 6, 1, 2, 1, 4, 5, 1, 3, 3, 1, 6, 5, 4, 2, 1, 5, 4, 6, 6, 1, 2, 4, 1, 4, 3, 2, 2, 1, 3, 4, 3, 6, 2, 1, 4, 3, 6, 2, 1, 2, 5, 2, 6, 1, 2, 3, 1, 4, 3, 5, 6, 1, 6, 5, 2, 4, 4, 4, 2, 1, 3, 3, 3, 3, 1, 3, 6, 6, 2, 4, 1, 2, 5, 2, 6, 1, 6, 3, 2, 1, 5, 4, 3, 4, 3, 1, 6, 2, 4, 1, 5, 2, 5, 3, 4, 6, 3, 2, 6, 3, 4, 6, 1, 5, 2, 5, 5, 5, 6, 5, 2, 5, 1, 4, 2, 3, 2, 5, 5, 6, 1, 1, 2, 4, 6, 3, 4, 5, 4, 4, 1, 4, 4, 1, 2, 6, 6, 6, 1, 4, 3, 3, 5, 6, 5, 2, 3, 4, 4, 6, 5, 1, 4, 3, 3, 1, 4, 4, 4, 4, 6, 6, 1, 4, 4, 1, 1, 3, 3, 4, 4, 4, 4, 4, 1, 3, 3, 4, 3, 3, 3, 5, 6, 1, 1, 1, 2, 6, 3, 1, 5, 5, 1, 5, 5, 3, 3, 5, 6, 2, 6, 2, 3, 1, 4, 6, 4, 2, 2, 5, 5, 2, 6, 5, 5, 5, 1, 4, 4, 4, 4, 5, 5, 5, 6, 6, 4, 6, 1, 2, 6, 6, 2, 5, 1, 6, 1, 3, 1, 1, 2, 3, 1, 1, 1, 1, 1, 2, 4, 6, 3, 4, 6, 6, 3, 4, 5, 4, 4, 2, 3, 2, 4, 1, 2, 3, 5, 3, 6, 3, 6, 4, 4, 6, 3, 5, 4, 1, 4, 5, 3, 3, 5, 4, 2, 6, 4, 4, 1, 4, 1, 1, 4, 4, 6, 3, 6, 3, 2, 4, 6, 1, 3, 6, 6, 6, 3, 1, 5, 6, 6, 3, 6, 6, 5, 1, 3, 6, 1, 6, 2, 5, 4, 6, 1, 5, 5, 3, 4, 4, 6, 4, 1, 2, 6, 5, 6, 3, 6, 2, 1, 1, 2, 3, 4, 5, 4, 3, 6, 5, 6, 5, 4, 4, 1, 6, 6, 2, 1, 6, 1, 5, 2, 3, 6, 4, 1, 5, 5, 2, 2, 2, 3, 3, 4, 6, 2, 4, 2, 1, 2, 6, 1, 3, 3, 1, 6, 4, 1, 4, 6, 3, 2, 1, 4, 2, 2, 1, 1, 1, 2, 4, 5, 1, 5, 5, 4, 5, 6, 3, 1, 5, 1, 1, 5, 4, 2, 2, 2, 5, 3, 6, 6, 3, 5, 4, 4, 1, 2, 4, 5, 6, 4, 5, 1, 6, 2, 2, 2, 5, 5, 2, 3, 6, 5, 6, 6, 2, 2, 1, 1, 5, 6, 4, 3, 2, 2, 3, 2, 5, 1, 1, 2, 5, 1, 4, 6, 2, 1, 2, 2, 4, 4, 1, 6, 2, 6, 3, 5, 3, 1, 3, 5, 2, 4, 6, 3, 5, 5, 2, 1, 3, 3, 5, 3, 4, 5, 1, 2, 2, 2, 6, 6, 1, 2, 6, 1, 1, 3, 2, 3, 2, 1, 6, 2, 6, 5, 6, 3, 1, 5, 3, 2, 6, 5, 6, 4, 3, 1, 1, 1, 3, 4, 2, 3, 4, 1, 5, 4, 3, 1, 5, 6, 4, 3, 6, 5, 2, 1, 2, 5, 2, 1, 2, 2, 2, 2, 5, 4, 2, 2, 3, 3, 5, 4, 3, 2, 3, 1, 4, 4, 6, 3, 5, 1, 1, 2, 6, 3, 2, 4, 5, 1, 5, 4, 5, 5, 4, 1, 1, 5, 6, 6, 5, 6, 5, 5, 2, 5, 1, 1, 1, 2, 5, 5, 5, 1, 5, 6, 2, 4, 5, 5, 6, 6, 6, 3, 2, 2, 4, 2, 5, 5, 4, 4, 5, 4, 2, 5, 1, 1, 6, 1, 4, 6, 6, 1, 4, 6, 2, 4, 5, 6, 3, 6, 3, 4, 2, 4, 4, 2, 6, 5, 5, 1, 2, 5, 6, 1, 5, 3, 2, 5, 3, 6, 2, 2, 4, 1, 4, 4, 5, 4, 3, 1, 6 \newline
	\textbf{Häufigkeiten:}
	\begin{center}
		\begin{tabular}{ | l | l | l | }
			\hline
			Augenzahl & Anzahl der Würfe & relative Häufigkeit \\ \hline
			1 & 164 & 0.164 \\ \hline
			2 & 177 & 0.177 \\ \hline
			3 & 158 & 0.158 \\ \hline
			4 & 164 & 0.164 \\ \hline
			5 & 175 & 0.175 \\ \hline
			6 & 162 & 0.162 \\ \hline
		\end{tabular}  
	\end{center}
	\textbf{Berechnung des Medians:} \newline
	$0.164 + 0.177 + 0.158 = 0.499 \leq 0.5$ \newline
	$0.164 + 0.177 + 0.158 + 0.164 = 0.663 > 0.5$ \newline
	Der Median ist also 3 (der dritte Summand stellt Augenzahl 3 da). \newline
	\textbf{Modalwert:} \newline
	$0.177 > 0.175 > 0.164 > 0.162 > 0.158$ \newline
	Der Modalwert ist also 2. \newline
	\textbf{Mittwelwert:} \newline 
	$m_1 = \sum_{i=0}^{n-1} x_i p_i$ \newline
	$m_1 = 1*0.164 + 2*0.177 + 3*0.158 + 4*0.164 + 5*0.175 + 6*0.162 \approx 3.5$ \newline
	\textbf{Standardabweichung:} \newline
	$s = \sqrt{\sum_{i=0}^{n-1} (x_i-m_1)^2 p_i}$ \newline
	$s = \sqrt{(1-3.495)^2*0.164 + (2-3.495)^2*0.177 + (3-3.495)^2*0.158} \\\overline{+ (4-3.495)^2*0.164 + (5-3.495)^2*0.175 + (6-3.495)^2*0.162} \approx 1.7$ \newpage
	\begin{figure}
		\includegraphics[width=1.0\textwidth]{A03_histo.png}
		\caption{Histogramm für 1000 Würfe mit einem 6-seitigen Würfel. \newline gelb: Standardabweichung, rot: Modalwert, grün: Median, violett: Mittwelwert}
	\end{figure}
	\textbf{Entropie:} \newline
	$H = \sum_{i=0}^{n-1} p_i log_2(\frac{1}{p_i})$ \newline
	$H = 0.164*log_2(\frac{1}{0.164}) + 0.177*log_2(\frac{1}{0.177}) + 0.158*log_2(\frac{1}{0.158}) + 0.164*log_2(\frac{1}{0.164}) + 0.175*log_2(\frac{1}{0.175}) + 0.162*log_2(\frac{1}{0.162}) \approx 2.58373$ \newline
	\textbf{Max. Entropie:} \newline
	$H_{max} = log_2(n)$ \newline
	$H_{max} = log_2(6) \approx 2.58496$ \newline
	\textbf{Redundanz:} \newline
	$R = H_{max} - H = 2.58496 - 2.58373 \approx 0.001$ \newline
	Die Redundanz ist sehr klein.
\end{document}
	\newpage
	\documentclass[fleqn,a4paper,12pt]{article}
\usepackage{standalone}		% Zum Einlesen aus anderen .tex-Files
\usepackage{geometry}		% Zur Bearbeitung des Layouts (Ränder,...)
\geometry{left=30mm, right=40mm, bottom=30mm}
\usepackage[german]{babel}
\usepackage[utf8]{inputenc}
\usepackage{amsmath}		% Mathematische Symbole
\usepackage{amssymb}     	% Nochmehr mathematische Symbole
\usepackage{dsfont}      	% Schriftsatz fuer Zahlenmengensymbole
%\usepackage{verbatim}   	% erweiterte Verbatim-Umgebung
\usepackage{alltt}       	% Quasi-Verbatim-Umgebung
\usepackage{fancyhdr}    	% Eigene Kopfzeilen
\usepackage{graphicx}    	% Zum Einbinden von Grafiken
% Einbinden einer eps-Grafik geht so: includegraphics{path}
\usepackage{wrapfig}
\usepackage{lscape}
\usepackage{rotating}
\usepackage{epstopdf}

% Skalierung der Grafiken
\setlength{\unitlength}{1cm}

\pagestyle{fancy}            % Eigene Kopfzeilen verwenden
\frenchspacing               % Kein Extrafreiraum nach Satzzeichen
\setlength{\parindent}{0pt}  % Neue Absaetze nicht einruecken
%\sloppy                     % Schlampige Absatzformatierung
\fussy                       % Penible Absatzformatierung
\linespread{1.5}             % Zeilenabstand

%andere Definitionen
\newcommand{\R}{{\mathbb R}}
\newcommand{\N}{{\mathbb N}}
\newcommand{\Z}{{\mathbb Z}}
\newcommand{\Q}{{\mathbb Q}}
\newcommand{\C}{{\mathbb C}}
\newcommand{\F}{\mathcal{F}}
\newcommand{\less}{\setminus}
\newcommand{\inv}{{}^{-1}}
\newcommand{\Land}{\bigwedge}
\newcommand{\Lor}{\bigvee}

\begin{document}
Übungsaufgabe 4: \newline
Messreihe Summe der Augenzahlen bei 1000 Würfe mit zwei Würfeln: \newline
9, 9, 9, 7, 7, 6, 5, 8, 7, 8, 10, 8, 7, 2, 4, 11, 6, 7, 7, 4, 6, 8, 7, 6, 12, 6, 4, 5, 7, 3, 5, 8, 6, 8, 7, 6, 6, 9, 3, 8, 7, 6, 6, 10, 11, 12, 4, 7, 7, 8, 9, 4, 5, 10, 3, 11, 8, 8, 7, 9, 8, 7, 11, 2, 8, 5, 8, 6, 9, 11, 8, 9, 11, 8, 6, 8, 5, 8, 5, 11, 4, 7, 7, 3, 3, 10, 10, 7, 8, 5, 8, 6, 8, 3, 12, 5, 7, 4, 5, 9, 8, 6, 7, 11, 7, 7, 3, 9, 12, 5, 9, 6, 5, 8, 3, 10, 7, 11, 7, 4, 11, 3, 9, 4, 7,10, 2, 11, 4, 5, 10, 7, 3, 7, 7, 10, 11, 7, 4, 7, 3, 9, 6, 7, 8, 6, 7, 6, 8, 7, 10, 6, 4, 8, 9, 6, 7, 11, 11, 9, 8, 6, 6, 11, 9, 9, 4, 8, 5, 4, 4, 4, 7, 7, 8, 10, 5, 7, 9, 9, 7, 5, 8, 5, 8, 8, 3, 7, 5, 7, 4, 7, 8, 8, 5, 6, 4, 5, 10, 9, 4, 5, 3, 6, 5, 7, 7, 11, 5, 7, 8, 8, 5, 5, 6, 7, 6, 6, 7, 5, 9, 11, 8, 6, 6, 11, 8, 5, 8, 6, 3, 5, 3, 3, 10, 11, 8, 5, 12, 9, 11, 7, 10, 9, 7, 6, 11, 9, 7, 3, 8, 7, 2, 6, 6, 7, 4, 10, 5, 8, 7, 6, 4, 6, 9, 4, 9, 6, 10, 4, 5, 10, 11, 8, 4, 6, 10, 5, 3, 7, 8, 2, 8, 9, 6, 5, 11, 6, 5, 11, 7, 8, 6, 12, 6, 6, 10, 4, 6, 9, 10, 6, 4, 7, 7, 8, 10, 10, 4, 3, 9, 9, 9, 3, 12, 4, 12, 8, 9, 12, 10, 11, 3, 8, 8, 5, 6, 8, 10, 7, 3, 11, 5, 7, 8, 5, 5, 6, 6, 7, 7, 10, 6, 8, 4, 4, 8, 12, 5, 7, 5, 6, 4, 7, 3, 9, 3, 11, 7, 8, 4, 7, 7, 4, 5, 9, 8, 8, 2, 6, 5, 4, 6, 5, 5,8, 9, 8, 11, 5, 9, 5, 8, 6, 8, 7, 3, 8, 7, 6, 6, 7, 4, 8, 7, 8, 10, 5, 5, 7, 8, 6, 4, 6, 8, 6, 6, 9, 7, 9, 5, 11, 10, 5, 7, 9, 8, 6, 12, 6, 10, 9, 3, 4, 5, 7, 7, 7, 4, 5, 7, 3, 9, 6, 6, 8, 11, 7, 8, 7, 7, 4, 2, 9, 5, 4, 6, 5, 4, 6, 4, 5, 5, 8, 11, 4, 9, 4, 5, 9, 8, 7, 6, 12, 8, 6, 9, 3, 8, 7, 6, 5, 7, 8, 5, 6, 7, 9, 5, 5, 11, 6, 5, 8, 8, 4, 4, 9, 6, 7, 8, 12, 8, 8, 8, 5, 7, 7, 7, 12, 5, 6, 10, 12, 9, 2, 5, 8, 6, 4, 4, 4, 6, 9, 8, 2, 4, 6, 8, 8, 2, 9, 9, 8, 7, 3, 5, 9, 6, 7, 10, 9, 9, 6, 3, 8, 8, 6, 6, 10, 6, 6, 7, 5, 7, 8, 10, 7, 7, 11, 9, 10, 8, 10, 9, 9, 6, 3, 4, 6, 5, 7, 10, 3, 6, 7,8, 3, 10, 7, 11, 7, 9, 10, 8, 9, 5, 6, 9, 8, 8, 6, 5, 9, 6, 2, 4, 11, 5, 11, 10, 7, 6, 6, 10, 4, 6, 6, 3, 5, 12, 4, 4, 11, 2, 8, 9, 7, 8, 9, 8, 7, 6, 3, 8, 9, 8, 9, 8, 7, 4, 10, 6, 4, 5, 6, 5, 7, 5, 6, 8, 10, 5, 3, 8, 11, 4, 6, 7, 7, 2, 8, 8, 10, 7, 4, 11, 6, 4, 5, 4, 10, 5, 8, 7, 7, 4, 6, 7, 6, 7, 8, 6, 6, 3, 9, 4, 3, 7, 10, 12, 10, 5, 6, 8, 9, 9, 7, 4, 5, 9, 8, 3, 9, 7, 3, 6, 6, 6, 6, 8, 7,2, 12, 6, 5, 7, 8, 8, 10, 7, 5, 6, 8, 5, 3, 10, 7, 6, 8, 11, 9, 10, 8, 7, 3, 6, 8, 4, 6, 6, 3, 4, 11, 3, 10, 6, 3, 3, 12, 9, 7, 5, 7, 3, 6, 8, 5, 3, 5, 7, 9, 2, 5, 10, 8, 8, 10, 9, 12, 5, 8, 7, 7, 9, 5, 9, 10, 12, 7, 8, 4, 10, 12, 3, 7, 9, 4, 5, 12, 10, 7, 9, 12, 9, 5, 10, 2, 9, 6, 9, 4, 5, 5, 9, 7, 5, 4, 9, 7, 5, 11, 6, 4, 11, 6, 7, 7, 8, 6, 7, 4, 6, 11, 4, 5, 8, 4, 5, 7, 4, 4, 7, 7, 2, 2, 7, 11, 9, 6, 5, 6, 6, 6, 9, 7, 4, 8, 6, 4, 10, 7, 6, 6, 4, 3, 6, 5, 9, 8, 2, 9, 8, 2, 5, 6, 7, 9, 4, 3, 3, 5, 3, 4, 2, 7, 4, 2, 9, 6, 5, 5, 7, 4, 5, 10, 2, 8, 5, 5, 8, 6, 4, 8, 4, 7, 6, 7, 7, 5, 4, 2, 10, 8, 7, 4, 7, 8, 8, 9, 7, 8, 3, 4, 10, 2, 8, 12, 6, 11, 10, 7, 9, 7, 4, 6, 12, 3, 6, 7, 7, 3, 9, 6, 6, 6, 10, 6, 5, 6, 4, 3, 6, 5, 7, 5, 4, 4, 6, 8, 9, 10, 10, 7, 10, 5, 6, 12, 5, 5, 7, 9, 9, 7,11, 6, 8, 6, 7, 7, 6, 6, 12, 10, 12, 7, 6, 4, 4, 3, 8, 6, 11, 4, 2, 4, 7, 10, 10, 5, 7, 5, 9, 11, 8, 3, 12, 5, 8, 6, 7, 6, 7, 8, 9, 11, 5, 4, 6, 8, 5, 6, 6, 6, 10, 7, 12, 7, 9, 7 \newline
\textbf{Häufigkeiten}
\begin{center}
	\begin{tabular}{ | l | l | l | }
		\hline
		Summe der Augenzahlen & Anzahl der Würfe & relative Häufigkeit \\ \hline
		2 & 26 & 0.026 \\ \hline
		3 & 59 & 0.059 \\ \hline
		4 & 97 & 0.097 \\ \hline
		5 & 118 & 0.118 \\ \hline
		6 & 153 & 0.153 \\ \hline
		7 & 162 & 0.162 \\ \hline
		8 & 138 & 0.138 \\ \hline
		9 & 98 & 0.098 \\ \hline
		10 & 68 & 0.068 \\ \hline
		11 & 50 & 0.05 \\ \hline
		12 & 31 & 0.031 \\ \hline
	\end{tabular}  
\end{center}
\textbf{Median:} 7 \newline
\textbf{Modalwert:} 7 \newline
\textbf{Mittwelwert:} 6.847 \newline
\textbf{Standardabweichung:} $\approx 2.4$ \newline
\textbf{Entropie:} $\approx 3.2658$ \newline
\textbf{Max. Entropie:} $\approx 3.4594$ \newline
\textbf{Redundanz:} $\approx 0.194$ \newline
\begin{figure}
	\includegraphics[width=1.0\textwidth]{A04_histo1.png}
	\caption{Normales Histogramm für die Summe der Augenzahlen mit zwei Würfeln bei 1000 Würfen}
\end{figure}
\begin{figure}
	\includegraphics[width=1.0\textwidth]{A04_histo2.png}
	\caption{Kumulative Histogramm für die Summe der Augenzahlen mit zwei Würfeln bei 1000 Würfen \newline gelb: Standardabweichung, rot: Modalwert, grün: Median, violett: Mittwelwert}
\end{figure}
\end{document}
	\newpage
	\documentclass[fleqn,a4paper,12pt]{article}

%used Packages
\usepackage{standalone}		% Zum Einlesen aus anderen .tex-Files
\usepackage{geometry}		% Zur Bearbeitung des Layouts (Ränder,...)
\usepackage[german]{babel}
\usepackage[utf8]{inputenc}
\usepackage{amsmath}		% Mathematische Symbole
\usepackage{amssymb}     	% Nochmehr mathematische Symbole
\usepackage{dsfont}      	% Schriftsatz fuer Zahlenmengensymbole
%\usepackage{verbatim}   	% erweiterte Verbatim-Umgebung
\usepackage{alltt}       	% Quasi-Verbatim-Umgebung
\usepackage{fancyhdr}    	% Eigene Kopfzeilen
\usepackage{graphicx}    	% Zum Einbinden von Grafiken
							% Einbinden einer eps-Grafik geht so: includegraphics{path}
\usepackage{wrapfig}
\usepackage{lscape}
\usepackage{rotating}
\usepackage{epstopdf}

% Skalierung der Grafiken
\setlength{\unitlength}{1cm}

\frenchspacing               % Kein Extrafreiraum nach Satzzeichen
\setlength{\parindent}{0pt}  % Neue Absaetze nicht einruecken
%\sloppy                     % Schlampige Absatzformatierung
\fussy                       % Penible Absatzformatierung
\linespread{1.5}             % Zeilenabstand


% Seitenraender
\geometry{left=30mm, right=40mm, bottom=30mm}
				% Doc-class, Packageimports, fancy stuff
%%Seitenränder formatieren
\addtolength{\voffset}{-2cm}
\addtolength{\textheight}{0cm}
\addtolength{\hoffset}{0cm}
\addtolength{\textwidth}{2cm}
\addtolength{\headheight}{2cm} % fuer jeden Strichkode einen Zentimeter

% Font fuer Code 39
\font\xlix=wlc39 scaled 1200
\newcommand\barcode[1]{{\xlix@#1@}}

% Name, Matrikelnummer, Barcode
\newcommand\student[2]{
	\mbox{\scriptsize
		\begin{tabular}{@{}l@{}r@{}}
			\multicolumn{2}{@{}r@{}}{\barcode{#2}}\\
			#1&#2\\
		\end{tabular}}}

% Kopfzeile
\pagestyle{fancy}            % Eigene Kopfzeilen verwenden
\lhead{
	\small
	\textsc{Grundlagen der Signalverarbeitung \\
		WS 2017/2018 \\
		\"Ubung (\today)}
	\vfill}
\rhead{
	\begin{tabular}[b]{@{}rr@{}}
		\student{Philipp Badenhoop}{572693} &
		\student{Steven Lange}{568733} \\
		\student{Pascal Jochmann}{575056} &
		\student{Kevin Trogant}{572451}
\end{tabular}}			% Definition der Kopfzeile
%andere Definitionen
\providecommand{\R}{{\mathbb R}}
\providecommand{\N}{{\mathbb N}}
\providecommand{\Z}{{\mathbb Z}}
\providecommand{\Q}{{\mathbb Q}}
\providecommand{\C}{{\mathbb C}}
\providecommand{\F}{\mathcal{F}}
\providecommand{\less}{\setminus}
\providecommand{\inv}{{}^{-1}}
\providecommand{\Land}{\bigwedge}
\providecommand{\Lor}{\bigvee}			% Liste der zusätzlichen Commands und redefines

\begin{document}
	Übungsaufgabe 5:\newline
	\begin{tabular}{l l}
		Funktion:		& $f(x) = -x \cdot \text{ld}(x) = -x\cdot{\log(x)}{\log(2)},\ x\in(0,1]$\\
		Domain:			& $\left\lbrace x\in\R\mid 0 < x \le 1\right\rbrace$\\
		Image:			& $\left\lbrace f(x)\in\R\mid 0 \le f(x) \le f\left(\frac{1}{e}\right) \right\rbrace$\\
		Graph:			& \\%\includegraphics{}\\
		1. Ableitung:	& $\frac{d}{dx}f(x) = $\\
		2. Ableitung:	& $\frac{d^2}{dx^2}f(x) = $\\
		3. Ableitung:	& $\frac{d^3}{dx^3}f(x) = $\\
		Stammfunktion:	& $F(x) = $???\\
		& $\int_0^1 f(x) dx = $\\
		Supremum:		& $f\left(\frac{1}{e}\right)$\\
		Maximum:		& $f\left(\frac{1}{e}\right)$\\
		Infimum:		& $0$\\
		Minimum:		& $f(1) = 0$\\
		Nullstellen:	& $x_0 = 1$\\
		Wendepunkt:		& -\\
		Scheitelpunkt:	& -\\
		Monotonie:		& $x\in\left(0,\frac{1}{e}\right)$ streng monoton wachsend\\
		& $x\in\left(\frac{1}{e},1\right]$ streng monoton fallend
		
	\end{tabular}
\end{document}
	\newpage
	\documentclass[fleqn,a4paper,12pt]{article}

%used Packages
\usepackage{standalone}		% Zum Einlesen aus anderen .tex-Files
\usepackage{geometry}		% Zur Bearbeitung des Layouts (Ränder,...)
\usepackage[german]{babel}
\usepackage[utf8]{inputenc}
\usepackage{amsmath}		% Mathematische Symbole
\usepackage{amssymb}     	% Nochmehr mathematische Symbole
\usepackage{dsfont}      	% Schriftsatz fuer Zahlenmengensymbole
%\usepackage{verbatim}   	% erweiterte Verbatim-Umgebung
\usepackage{alltt}       	% Quasi-Verbatim-Umgebung
\usepackage{fancyhdr}    	% Eigene Kopfzeilen
\usepackage{graphicx}    	% Zum Einbinden von Grafiken
							% Einbinden einer eps-Grafik geht so: includegraphics{path}
\usepackage{wrapfig}
\usepackage{lscape}
\usepackage{rotating}
\usepackage{epstopdf}

% Skalierung der Grafiken
\setlength{\unitlength}{1cm}

\frenchspacing               % Kein Extrafreiraum nach Satzzeichen
\setlength{\parindent}{0pt}  % Neue Absaetze nicht einruecken
%\sloppy                     % Schlampige Absatzformatierung
\fussy                       % Penible Absatzformatierung
\linespread{1.5}             % Zeilenabstand


% Seitenraender
\geometry{left=30mm, right=40mm, bottom=30mm}
				% Doc-class, Packageimports, fancy stuff
%%Seitenränder formatieren
\addtolength{\voffset}{-2cm}
\addtolength{\textheight}{0cm}
\addtolength{\hoffset}{0cm}
\addtolength{\textwidth}{2cm}
\addtolength{\headheight}{2cm} % fuer jeden Strichkode einen Zentimeter

% Font fuer Code 39
\font\xlix=wlc39 scaled 1200
\newcommand\barcode[1]{{\xlix@#1@}}

% Name, Matrikelnummer, Barcode
\newcommand\student[2]{
	\mbox{\scriptsize
		\begin{tabular}{@{}l@{}r@{}}
			\multicolumn{2}{@{}r@{}}{\barcode{#2}}\\
			#1&#2\\
		\end{tabular}}}

% Kopfzeile
\pagestyle{fancy}            % Eigene Kopfzeilen verwenden
\lhead{
	\small
	\textsc{Grundlagen der Signalverarbeitung \\
		WS 2017/2018 \\
		\"Ubung (\today)}
	\vfill}
\rhead{
	\begin{tabular}[b]{@{}rr@{}}
		\student{Philipp Badenhoop}{572693} &
		\student{Steven Lange}{568733} \\
		\student{Pascal Jochmann}{575056} &
		\student{Kevin Trogant}{572451}
\end{tabular}}			% Definition der Kopfzeile
%andere Definitionen
\providecommand{\R}{{\mathbb R}}
\providecommand{\N}{{\mathbb N}}
\providecommand{\Z}{{\mathbb Z}}
\providecommand{\Q}{{\mathbb Q}}
\providecommand{\C}{{\mathbb C}}
\providecommand{\F}{\mathcal{F}}
\providecommand{\less}{\setminus}
\providecommand{\inv}{{}^{-1}}
\providecommand{\Land}{\bigwedge}
\providecommand{\Lor}{\bigvee}			% Liste der zusätzlichen Commands und redefines

\begin{document}
	Übungsaufgabe 6: \newline
	Das Histogramm beschreibt eine Wahrscheinlichkeitsdichtefunktion. Die Fl\"ache unter der Funktion $p(x)$ ist gleich 1. \\
	Der Mittelwert ist $m = \frac{a+b}{2} = \frac{7}{2} = 3,5$ \\
	Der Varianz ist $v = \frac{(b-a)^2}{12} = \frac{36}{12} = 3 $ \\
	Die Standardabweichung ist $\sqrt{v} = 1,732$
	\begin{figure}
		\includegraphics[width=1.0\textwidth]{A06_Histo.png}
		\caption{normiertes Histogramm f\"ur runden W\"urfel}
	\end{figure}
	
\end{document}
\end{document}