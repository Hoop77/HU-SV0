\documentclass[fleqn,a4paper,12pt]{article}

%used Packages
\usepackage{standalone}		% Zum Einlesen aus anderen .tex-Files
\usepackage{geometry}		% Zur Bearbeitung des Layouts (Ränder,...)
\usepackage[german]{babel}
\usepackage[utf8]{inputenc}
\usepackage{amsmath}		% Mathematische Symbole
\usepackage{amssymb}     	% Nochmehr mathematische Symbole
\usepackage{dsfont}      	% Schriftsatz fuer Zahlenmengensymbole
%\usepackage{verbatim}   	% erweiterte Verbatim-Umgebung
\usepackage{alltt}       	% Quasi-Verbatim-Umgebung
\usepackage{fancyhdr}    	% Eigene Kopfzeilen
\usepackage{graphicx}    	% Zum Einbinden von Grafiken
							% Einbinden einer eps-Grafik geht so: includegraphics{path}
\usepackage{wrapfig}
\usepackage{lscape}
\usepackage{rotating}
\usepackage{epstopdf}

% Skalierung der Grafiken
\setlength{\unitlength}{1cm}

\frenchspacing               % Kein Extrafreiraum nach Satzzeichen
\setlength{\parindent}{0pt}  % Neue Absaetze nicht einruecken
%\sloppy                     % Schlampige Absatzformatierung
\fussy                       % Penible Absatzformatierung
\linespread{1.5}             % Zeilenabstand


% Seitenraender
\geometry{left=30mm, right=40mm, bottom=30mm}
				% Doc-class, Packageimports, fancy stuff
%%Seitenränder formatieren
\addtolength{\voffset}{-2cm}
\addtolength{\textheight}{0cm}
\addtolength{\hoffset}{0cm}
\addtolength{\textwidth}{2cm}
\addtolength{\headheight}{2cm} % fuer jeden Strichkode einen Zentimeter

% Font fuer Code 39
\font\xlix=wlc39 scaled 1200
\newcommand\barcode[1]{{\xlix@#1@}}

% Name, Matrikelnummer, Barcode
\newcommand\student[2]{
	\mbox{\scriptsize
		\begin{tabular}{@{}l@{}r@{}}
			\multicolumn{2}{@{}r@{}}{\barcode{#2}}\\
			#1&#2\\
		\end{tabular}}}

% Kopfzeile
\pagestyle{fancy}            % Eigene Kopfzeilen verwenden
\lhead{
	\small
	\textsc{Grundlagen der Signalverarbeitung \\
		WS 2017/2018 \\
		\"Ubung (\today)}
	\vfill}
\rhead{
	\begin{tabular}[b]{@{}rr@{}}
		\student{Philipp Badenhoop}{572693} &
		\student{Steven Lange}{568733} \\
		\student{Pascal Jochmann}{575056} &
		\student{Kevin Trogant}{572451}
\end{tabular}}			% Definition der Kopfzeile
%andere Definitionen
\providecommand{\R}{{\mathbb R}}
\providecommand{\N}{{\mathbb N}}
\providecommand{\Z}{{\mathbb Z}}
\providecommand{\Q}{{\mathbb Q}}
\providecommand{\C}{{\mathbb C}}
\providecommand{\F}{\mathcal{F}}
\providecommand{\less}{\setminus}
\providecommand{\inv}{{}^{-1}}
\providecommand{\Land}{\bigwedge}
\providecommand{\Lor}{\bigvee}			% Liste der zusätzlichen Commands und redefines

\begin{document}
	\section*{Übungsaufgabe 26:}
	Berechnete Koeffizienten mit dem Verfahren aus 1921:\newline
	\begin{center}
		\begin{tabular}{|c|c|c|}
			\hline 
			$k$ & $a_k$ & $b_k$ \\
			\hline
			0   &  128.55    & 0 \\
			1   &  77.9    & 62.6 \\
			2   &  1.3    & 1.9 \\
			3   &  -31.5    & -11.1 \\
			4   &  -1.9    & -1.4 \\
			5   &  0.5    & 1.4 \\
			6   &  0.3    & 11.6 \\
			7   &  -1    & 0 \\
			8   &  -0.5    & -0.5 \\
			9   &  0.9    & 1.2 \\
			\hline
		\end{tabular}
	\end{center}
	Berechnete Koeffizienten mit Matlab:
	\begin{center}
		\begin{tabular}{|c|c|c|}
			\hline 
			$k$ & $a_k$ & $b_k$ \\
			\hline
			0   &  257.1    & 0 \\
			1   &  78    & 62.6 \\
			2   &  1    & 1.76 \\
			3   &  -32    & -10.8 \\
			4   &  -1    & -1.35 \\
			5   &  0.5    & 1.4 \\
			6   &  0.28    & 1.15 \\
			7   &  -0.64    & -1.6 \\
			8   &  0.45    & -0.57 \\
			9   &  0.72    & 1.17 \\
			\hline
		\end{tabular}
	\end{center}
	An dieser Stelle sei gesagt, dass wir mit Matlab die Koeffizienten wiederum durch die Fouriersynthese überprüft haben.
	Im Vergleich sind die Werte bis mit 0 < k < 6 sehr ähnlich, wobei der Rest deutliche Abweichungen zeigt.l
	
\end{document}