\documentclass[fleqn,a4paper,12pt]{article}

%used Packages
\usepackage{standalone}		% Zum Einlesen aus anderen .tex-Files
\usepackage{geometry}		% Zur Bearbeitung des Layouts (Ränder,...)
\usepackage[german]{babel}
\usepackage[utf8]{inputenc}
\usepackage{amsmath}		% Mathematische Symbole
\usepackage{amssymb}     	% Nochmehr mathematische Symbole
\usepackage{dsfont}      	% Schriftsatz fuer Zahlenmengensymbole
%\usepackage{verbatim}   	% erweiterte Verbatim-Umgebung
\usepackage{alltt}       	% Quasi-Verbatim-Umgebung
\usepackage{fancyhdr}    	% Eigene Kopfzeilen
\usepackage{graphicx}    	% Zum Einbinden von Grafiken
							% Einbinden einer eps-Grafik geht so: includegraphics{path}
\usepackage{wrapfig}
\usepackage{lscape}
\usepackage{rotating}
\usepackage{epstopdf}

% Skalierung der Grafiken
\setlength{\unitlength}{1cm}

\frenchspacing               % Kein Extrafreiraum nach Satzzeichen
\setlength{\parindent}{0pt}  % Neue Absaetze nicht einruecken
%\sloppy                     % Schlampige Absatzformatierung
\fussy                       % Penible Absatzformatierung
\linespread{1.5}             % Zeilenabstand


% Seitenraender
\geometry{left=30mm, right=40mm, bottom=30mm}
				% Doc-class, Packageimports, fancy stuff
%Seitenränder formatieren
\addtolength{\voffset}{-2cm}
\addtolength{\textheight}{0cm}
\addtolength{\hoffset}{0cm}
\addtolength{\textwidth}{2cm}
\addtolength{\headheight}{2cm} % fuer jeden Strichkode einen Zentimeter

% Font fuer Code 39
\font\xlix=wlc39 scaled 1200
\newcommand\barcode[1]{{\xlix@#1@}}

% Name, Matrikelnummer, Barcode
\newcommand\student[2]{
	\mbox{\scriptsize
		\begin{tabular}{@{}l@{}r@{}}
			\multicolumn{2}{@{}r@{}}{\barcode{#2}}\\
			#1&#2\\
		\end{tabular}}}

% Kopfzeile
\pagestyle{fancy}            % Eigene Kopfzeilen verwenden
\lhead{
	\small
	\textsc{Grundlagen der Signalverarbeitung \\
		WS 2017/2018 \\
		\"Ubung (\today)}
	\vfill}
\rhead{
	\begin{tabular}[b]{@{}rr@{}}
		\student{Philipp Badenhoop}{572693} &
		\student{Steven Lange}{568733} \\
		\student{Pascal Jochmann}{575056} &
		\student{Kevin Trogant}{572451}
\end{tabular}}			% Definition der Kopfzeile
%andere Definitionen
\providecommand{\R}{{\mathbb R}}
\providecommand{\N}{{\mathbb N}}
\providecommand{\Z}{{\mathbb Z}}
\providecommand{\Q}{{\mathbb Q}}
\providecommand{\C}{{\mathbb C}}
\providecommand{\F}{\mathcal{F}}
\providecommand{\less}{\setminus}
\providecommand{\inv}{{}^{-1}}
\providecommand{\Land}{\bigwedge}
\providecommand{\Lor}{\bigvee}			% Liste der zusätzlichen Commands und redefines

\begin{document}
	\section*{Übungsaufgabe 35:}

    $
    \textbf{DFT} = \begin{bmatrix}
        0.5 & 0.5  & 0.5   &  0.5 \\
        0.5 & -0.5j & -0.5 &  0.5j \\
        0.5 & -0.5 & 0.5   & -0.5 \\
        0.5 & 0.5j & -0.5  & -0.5j
    \end{bmatrix}$

    Da ${\textbf{DFT}^T}^{*} = \textbf{DFT}^{-1}$ gilt und, weil $\textbf{DFT}$ symmetrisch ist, gilt:
    \[{\textbf{DFT}^T}^{*} = \textbf{DFT}^{*} = \begin{bmatrix}
        0.5 & 0.5   & 0.5  &  0.5 \\
        0.5 & 0.5j  & -0.5 & -0.5j \\
        0.5 & -0.5  & 0.5  & -0.5 \\
        0.5 & -0.5j & -0.5 & 0.5j
    \end{bmatrix} = \textbf{DFT}^{-1}\]

    $s = (1, 2, 3, 1)^T$ transformieren mit $\textbf{DFT}$ ergibt:
    \[
        \textbf{DFT}s = (3.5, -1 - 0.5j, 0.5, -1 + 0.5j)^T
    \]

    Zurücktransformation mittels $\textbf{DFT}^{-1}$ ergibt:
    \[
        \textbf{DFT}^{-1} (3.5, -1 - 0.5j, 0.5, -1 + 0.5j)^T = (1, 2, 3, 1)^T = s
    \]
    \newpage
	Realteilspektrum:\\
    \includegraphics[width=\textwidth]{re_spectrum.png}
    \newpage
    Imaginärteilspektrum:\\
    \includegraphics[width=\textwidth]{im_spectrum.png}
\newpage
	\section*{Übungsaufgabe 36:}
		\includegraphics[width = \textwidth]{A36_Zeit-App-Plot.png}
		Für die Frequenzauflösung gilt:
		$$\Delta f = \frac{1}{T_B}$$
		Für die Beobachtungsdauer $T_B = T_A\cdot N$ gilt mit $T_A = 0.005s$ und $N = 1000$: $T_B = 5$ und dem entsprechend $\Delta f = 0.2\text{Hz}$.\\
		\\
		\\
		Für den Vektorbetrag von $s\in\R^{1000}$ und $S\in\R^{1000}$ gilt 
		$$\|s\|_{2} \approx 93.68 \approx \|S\|_2 = \|DFT\cdot s\|_2$$
\newpage
	\section*{Übungsaufgabe 37:}

	$
    DHT =
    \begin{bmatrix}
    0.35 & 0.35 & 0.35 & 0.35 & 0.35 & 0.35 & 0.35 & 0.35 & \\
    0.35 & 0.35 & 0.35 & 0.35 & 0.35 & 0.35 & 0.35 & 0.35 & \\
    0.5 & 0.5 & 0.5 & 0.5 & -0.5 & -0.5 & -0.5 & -0.5 & \\
    0.71 & 0.71 & -0.71 & -0.71 & 0.0 & 0.0 & 0.0 & 0.0 & \\
    0.0 & 0.0 & 0.0 & 0.0 & 0.71 & 0.71 & -0.71 & -0.71 & \\
    1.0 & -1.0 & 0.0 & 0.0 & 0.0 & 0.0 & 0.0 & 0.0 & \\
    0.0 & 0.0 & 1.0 & -1.0 & 0.0 & 0.0 & 0.0 & 0.0 & \\
    0.0 & 0.0 & 0.0 & 0.0 & 1.0 & -1.0 & 0.0 & 0.0 & \\
    \end{bmatrix}
    $ \\ $
    s' =
    \begin{bmatrix}
    1.41 & 1.41 & 5.0 & -0.71 & -0.71 & -1.0 & 2.0 & 0.0 & \\
    \end{bmatrix}
    ^T $ \\ $
    s = DHT \cdot s' =
    \begin{bmatrix}
    2.0 & 4.0 & 6.0 & 2.0 & -2.0 & -2.0 & -1.0 & -1.0 & \\
    \end{bmatrix}
    ^T $ \newline

    Hier wurde der Algorithmus im Buch verwendet. Leider scheint bei der Rücktransformation nicht wieder das originale Signal herauszukommen.
    Dies ist unsere Implementation in Python: \newline
    \lstinputlisting[language=Python]{A37.py}



\newpage
	\section*{Übungsaufgabe 38:}

	$
    DHT =
    \begin{bmatrix}
	0.3536 &    0.3536 &    0.3536 &   0.3536 &    0.3536 &   0.3536 &    0.3536 &    0.3536 \\
    0.3536 &    0.3536 &     0.3536 &   0.3536 &   -0.3536 &  -0.3536 &   -0.3536 &    -0.3536 \\
    0.5000 &    0.5000 &  -0.5000 &  -0.5000 &        0 &        0 &        0 &        0 \\
         0 &        0 &        0 &        0 &   0.5000 &   0.5000 &  -0.5000 &  -0.5000 \\
    0.7071 &  -0.7071 &        0 &        0 &        0 &        0 &        0 &        0 \\
         0 &        0 &   0.7071 &   -0.7071 &        0 &        0 &        0 &        0 \\
         0 &        0 &        0  &       0 &   0.7071 &  -0.7071 &        0 &        0 \\
         0 &        0 &        0   &      0 &        0 &        0 &   0.7071 &  -0.7071 \\

    \end{bmatrix}
    $ \\ $
    s' =
    \begin{bmatrix}
		1.4142 &   3.5355 &  -0.5000 &  -0.5000 &  -0.7071 &   1.4142 &        0 &  -0.7071 \\
    \end{bmatrix}
    ^T $ \\ $
    s = DHT \cdot s' =
    \begin{bmatrix}
    1.0 & 2.0 & 3.0 & 1.0 & -1.0 & -1.0 & -1.0 & 0.0 & \\
    \end{bmatrix}
    ^T $ \newline
	
\newpage\end{document}