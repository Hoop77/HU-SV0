\documentclass[fleqn,a4paper,12pt]{article}

%used Packages
\usepackage{standalone}		% Zum Einlesen aus anderen .tex-Files
\usepackage{geometry}		% Zur Bearbeitung des Layouts (Ränder,...)
\usepackage[german]{babel}
\usepackage[utf8]{inputenc}
\usepackage{amsmath}		% Mathematische Symbole
\usepackage{amssymb}     	% Nochmehr mathematische Symbole
\usepackage{dsfont}      	% Schriftsatz fuer Zahlenmengensymbole
%\usepackage{verbatim}   	% erweiterte Verbatim-Umgebung
\usepackage{alltt}       	% Quasi-Verbatim-Umgebung
\usepackage{fancyhdr}    	% Eigene Kopfzeilen
\usepackage{graphicx}    	% Zum Einbinden von Grafiken
							% Einbinden einer eps-Grafik geht so: includegraphics{path}
\usepackage{wrapfig}
\usepackage{lscape}
\usepackage{rotating}
\usepackage{epstopdf}

% Skalierung der Grafiken
\setlength{\unitlength}{1cm}

\frenchspacing               % Kein Extrafreiraum nach Satzzeichen
\setlength{\parindent}{0pt}  % Neue Absaetze nicht einruecken
%\sloppy                     % Schlampige Absatzformatierung
\fussy                       % Penible Absatzformatierung
\linespread{1.5}             % Zeilenabstand


% Seitenraender
\geometry{left=30mm, right=40mm, bottom=30mm}
